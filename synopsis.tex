\documentclass[paper=a4, fontsize=11pt]{scrartcl} % A4 paper and 11pt font size

\usepackage[T1]{fontenc} % Use 8-bit encoding that has 256 glyphs
\usepackage{fourier} % Use the Adobe Utopia font for the document - comment this line to return to the LaTeX default
\usepackage[english]{babel} % English language/hyphenation
\usepackage{amsmath,amsfonts,amsthm} % Math packages

\usepackage{lipsum} % Used for inserting dummy 'Lorem ipsum' text into the template

\usepackage{sectsty} % Allows customizing section commands
\allsectionsfont{\centering \normalfont\scshape} % Make all sections centered, the default font and small caps

\usepackage{fancyhdr} % Custom headers and footers
\pagestyle{fancyplain} % Makes all pages in the document conform to the custom headers and footers
\fancyhead{} % No page header - if you want one, create it in the same way as the footers below
\fancyfoot[L]{} % Empty left footer
\fancyfoot[C]{} % Empty center footer
\fancyfoot[R]{\thepage} % Page numbering for right footer
\renewcommand{\headrulewidth}{0pt} % Remove header underlines
\renewcommand{\footrulewidth}{0pt} % Remove footer underlines
\setlength{\headheight}{13.6pt} % Customize the height of the header

\numberwithin{equation}{section} % Number equations within sections (i.e. 1.1, 1.2, 2.1, 2.2 instead of 1, 2, 3, 4)
\numberwithin{figure}{section} % Number figures within sections (i.e. 1.1, 1.2, 2.1, 2.2 instead of 1, 2, 3, 4)
\numberwithin{table}{section} % Number tables within sections (i.e. 1.1, 1.2, 2.1, 2.2 instead of 1, 2, 3, 4)

\setlength\parindent{0pt} % Removes all indentation from paragraphs - comment this line for an assignment with lots of text

%----------------------------------------------------------------------------------------
%	TITLE SECTION
%----------------------------------------------------------------------------------------

\newcommand{\horrule}[1]{\rule{\linewidth}{#1}} % Create horizontal rule command with 1 argument of height

\title{	
\normalfont \normalsize 
\textsc{NSIT, Division of Computer Engineering} \\ [25pt] % Your university, school and/or department name(s)
\horrule{0.5pt} \\[0.4cm] % Thin top horizontal rule
\huge Project Synopsis \\ % The assignment title
\horrule{2pt} \\[0.5cm] % Thick bottom horizontal rule
}

\author{
		\horrule{0.5pt} \\[0.4cm]
		Ashish Kothari (240/CO/12)\\
		Ayush Gupta (248/CO/12)\\
		Chandan Kar (252/CO/12)\\
		\horrule{0.5pt} \\[0.4cm]} % Your name
\date{\normalsize\today} % Today's date or a custom date

\begin{document}

\maketitle % Print the title

%----------------------------------------------------------------------------------------
%	PROBLEM 1
%----------------------------------------------------------------------------------------

\section{Fast \& light console-based Music Player}

This project aims to build a fast, light and high-quality music player that is console-based with a simple and easy-to-use GUI written on shell scripts made for Linux systems. The music player will be able to decode most modern extensions of audio files.

%------------------------------------------------

\subsection{Utility Requirements}

The utility requires very little computing power to function and is very efficient with CPU usage. For GNU/Linux OS, the utility has about 3\% to 4\% CPU usage on playing a high quality VBR MP3 on a Pentium 366MHz machine.\\

Requirements are listed as below:
\begin{itemize}
	\item mpg123 - MPEG player and decoder library
	\item Linux OS
	\item Atleast 100MHz of computing power 
\end{itemize}


%------------------------------------------------

\subsection{Innovativeness \& Usefulness}

The music player is being developed keeping in mind minimal usage of computing power while having an easy-to-use GUI to play high quality music. \\

Some proposed features of the utility are:
\begin{itemize} \itemsep 1pt

	\item Support for various audio subsystems
	\item Simple but powerful control modes for frontend GUI
	\item Support for playback through URL
	\item Support for music playback on Cloud
	\item Many audio data settings: resampling, choose channel, mono, ...
	\item Support for Relative Volume Adjustment / Replay Gain

\end{itemize}

%------------------------------------------------

\subsection {Current status of development}
\begin{itemize} \itemsep -2pt

	\item Basic functionality of music player achieved.
	\item Frontend GUI under development and is being developed using Zenity library for shell scripts.
	\item Integration of proposed features into the GUI of the utility left.  

\end{itemize}

%------------------------------------------------

% \paragraph{Heading on level 4 (paragraph)}

% \lipsum[6] % Dummy text

% %----------------------------------------------------------------------------------------
% %	PROBLEM 2
% %----------------------------------------------------------------------------------------

% \section{Lists}

% %------------------------------------------------

% \subsection{Example of list (3*itemize)}
% \begin{itemize}
% 	\item First item in a list 
% 		\begin{itemize}
% 		\item First item in a list 
% 			\begin{itemize}
% 			\item First item in a list 
% 			\item Second item in a list 
% 			\end{itemize}
% 		\item Second item in a list 
% 		\end{itemize}
% 	\item Second item in a list 
% \end{itemize}

% %------------------------------------------------

% \subsection{Example of list (enumerate)}
% \begin{enumerate}
% \item First item in a list 
% \item Second item in a list 
% \item Third item in a list
% \end{enumerate}

%----------------------------------------------------------------------------------------

\end{document}